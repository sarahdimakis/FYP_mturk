\documentclass[man,draftall]{apa6}
\usepackage{lmodern}
\usepackage{amssymb,amsmath}
\usepackage{ifxetex,ifluatex}
\usepackage{fixltx2e} % provides \textsubscript
\ifnum 0\ifxetex 1\fi\ifluatex 1\fi=0 % if pdftex
  \usepackage[T1]{fontenc}
  \usepackage[utf8]{inputenc}
\else % if luatex or xelatex
  \ifxetex
    \usepackage{mathspec}
  \else
    \usepackage{fontspec}
  \fi
  \defaultfontfeatures{Ligatures=TeX,Scale=MatchLowercase}
\fi
% use upquote if available, for straight quotes in verbatim environments
\IfFileExists{upquote.sty}{\usepackage{upquote}}{}
% use microtype if available
\IfFileExists{microtype.sty}{%
\usepackage{microtype}
\UseMicrotypeSet[protrusion]{basicmath} % disable protrusion for tt fonts
}{}
\usepackage{hyperref}
\hypersetup{unicode=true,
            pdftitle={The different effect of anger and fear on political judgments},
            pdfauthor={Sarah Dimakis},
            pdfborder={0 0 0},
            breaklinks=true}
\urlstyle{same}  % don't use monospace font for urls
\usepackage{graphicx,grffile}
\makeatletter
\def\maxwidth{\ifdim\Gin@nat@width>\linewidth\linewidth\else\Gin@nat@width\fi}
\def\maxheight{\ifdim\Gin@nat@height>\textheight\textheight\else\Gin@nat@height\fi}
\makeatother
% Scale images if necessary, so that they will not overflow the page
% margins by default, and it is still possible to overwrite the defaults
% using explicit options in \includegraphics[width, height, ...]{}
\setkeys{Gin}{width=\maxwidth,height=\maxheight,keepaspectratio}
\IfFileExists{parskip.sty}{%
\usepackage{parskip}
}{% else
\setlength{\parindent}{0pt}
\setlength{\parskip}{6pt plus 2pt minus 1pt}
}
\setlength{\emergencystretch}{3em}  % prevent overfull lines
\providecommand{\tightlist}{%
  \setlength{\itemsep}{0pt}\setlength{\parskip}{0pt}}
\setcounter{secnumdepth}{0}
% Redefines (sub)paragraphs to behave more like sections
\ifx\paragraph\undefined\else
\let\oldparagraph\paragraph
\renewcommand{\paragraph}[1]{\oldparagraph{#1}\mbox{}}
\fi
\ifx\subparagraph\undefined\else
\let\oldsubparagraph\subparagraph
\renewcommand{\subparagraph}[1]{\oldsubparagraph{#1}\mbox{}}
\fi

%%% Use protect on footnotes to avoid problems with footnotes in titles
\let\rmarkdownfootnote\footnote%
\def\footnote{\protect\rmarkdownfootnote}


  \title{The different effect of anger and fear on political judgments}
    \author{Sarah Dimakis\textsuperscript{1}}
    \date{}
  
\shorttitle{ANGER AND FEAR ON POLITICAL JUDGMENTS}
\affiliation{
\vspace{0.5cm}
\textsuperscript{1} Univeristy of Oregon}
\usepackage{csquotes}
\usepackage{upgreek}
\captionsetup{font=singlespacing,justification=justified}

\usepackage{longtable}
\usepackage{lscape}
\usepackage{multirow}
\usepackage{tabularx}
\usepackage[flushleft]{threeparttable}
\usepackage{threeparttablex}

\newenvironment{lltable}{\begin{landscape}\begin{center}\begin{ThreePartTable}}{\end{ThreePartTable}\end{center}\end{landscape}}

\makeatletter
\newcommand\LastLTentrywidth{1em}
\newlength\longtablewidth
\setlength{\longtablewidth}{1in}
\newcommand{\getlongtablewidth}{\begingroup \ifcsname LT@\roman{LT@tables}\endcsname \global\longtablewidth=0pt \renewcommand{\LT@entry}[2]{\global\advance\longtablewidth by ##2\relax\gdef\LastLTentrywidth{##2}}\@nameuse{LT@\roman{LT@tables}} \fi \endgroup}


\DeclareDelayedFloatFlavor{ThreePartTable}{table}
\DeclareDelayedFloatFlavor{lltable}{table}
\DeclareDelayedFloatFlavor*{longtable}{table}
\makeatletter
\renewcommand{\efloat@iwrite}[1]{\immediate\expandafter\protected@write\csname efloat@post#1\endcsname{}}
\makeatother

\authornote{

Correspondence concerning this article should be addressed to Sarah
Dimakis, University of Oregon Department of Psychology. E-mail:
\href{mailto:sdimakis@uoregon.edu}{\nolinkurl{sdimakis@uoregon.edu}}}

\abstract{
Insert abstract here.


}

\begin{document}
\maketitle

\section{Method}\label{method}

\subsection{Participants}\label{participants}

Two-hundred U.S. adults were recruited via Amazon Mechanical Turk to
participate in a study listed as \enquote{Psychology and society
survey.} Participants were removed for completing the survey quicker
than predetermined time cutoffs or failing to pass attention checks,
leaving 134 participants for analyses. Of the remaining 134
participants, who were 20 to 68 years old (\emph{M} = 40.35, \emph{SD} =
11.99), 53.7\% identified as female and 46.3\% identified as male, while
77.6\% identified as White of Caucasian, 7.5\% as African American or
Black, 6\% as Hispanic or Latinx, 5.2\% as Asian or Asian American,
0.7\% as Native American, and 9.7\% as \enquote{Other.} The sample
leaned liberal, with 44\% affiliated with the Democratic Party, 23.1\%
affiliated with the Republican Party, and 9.7\% affiliated with the
Green Party, Libertarian Party, or no party. The participants were
compensated \$1.50 for completing the survey, which took on average less
than 12 minutes.

\subsection{Materials}\label{materials}

\textbf{Emotion induction.} Affect was induced incidentally through
video clips (Schaefer, Nils, Sanchex, \& Philippot, 2010). Six videos
were played for participants from three affect categories in order to
increase generalizability. Those in the control condition saw one of two
clips from Blue, a typical car ride or a man shuffling papers around in
an office. The two clips intending to elicit primarily fear were from
the Shining (i.e., the protagonists are escaping an axe murderer) and
the Blair Witch Project (i.e., the protagonists are frantically
searching through a house in the dark), while the clips intending to
elicit primarily anger were from Seven (i.e., a serial killer torments a
cop by murdering his wife) and Schindler's List (i.e., a Nazi murders in
a concentration camp).\\
\textbf{Issue and policy.} The participants read a short speech about
the prevalence of homelessness in the United States (see Appendix A).
The speech presented facts about the increasing number of homeless
people in the country, reasoning that the problem is important and worth
addressing. The participants also read a policy addressing homelessness
(see Appendix A). The policy was either complex or simple, and either
conservative or liberal leaning. The complex policies were comprised of
simple policies, such that a liberal, complex policy proposed three
simple policies that would be implemented concurrently. Half of the
policies were taken from prominent liberal websites and half were taken
from prominent conservative websites.\\
\textbf{Policy and issue judgments.} Participants rated ten items on a
scale of 1 = \enquote{strongly disagree} to 7 = \enquote{strongly
agree.} The items were judgments about the policy, such as \enquote{The
policy will solve the problem,} and judgments about the issue itself,
such as \enquote{Homelessness is an important issue} (see Appendix B).
The items mainly covered varying facets of policy beliefs, including if
the policy was good, reasonable, possible, effective, and moral.\\
\textbf{Individual difference measures.} Participants filled out several
individual difference surveys. An adjusted Positive and Negative Affect
Schedule (Watson, Clark, \& Tellegen, 1988) measured the extent to which
participants felt interested, distressed, afraid, guilty, angry,
disgusted, happy, sad, anxious, calm, surprised, and upset. The 12-point
Social and Economic Conservatism Scale (Everett, 2013) measured the
extent to which participants felt positively about conservative issues,
such as gun ownership or limited government. The 10-item Personality
Inventory (Gosling, Rentfrow, \& Swann, 2003) measured the personality
traits extraversion, agreeableness, conscientiousness, emotional
stability, and openness to experiences. The 18-point Need for Cognition
Scale (Cacioppo, Petty, \& Kao, 1984) measured the tendency and
enthusiasm for thinking complexly and abstractly. And last, the 10-point
Emotional Regulation Questionnaire (Gross \& John, 2003) measured the
propensity to regulate emotions through cognitive reappraisal and
expressive suppression techniques. Gender, age, ethnicity, political
orientation, and political affiliation were also collected. \#\#
Procedure Participants were redirected to complete the online survey
through the Amazon Mechanical Turk website. Two thirds of participants
were randomly assigned to watch an emotionally evocative film clip, and
the remaining third were randomly assigned to the control condition,
where they watched a neutral film clip. The emotionally evocative clips
had been reliability found to induce predominantly fear or predominantly
anger. After watching a film clip, participants read a short speech
about the prevalance of homelessness in the United States. Then,
participants were randomly assigned to read a policy addressing
homelessness that was either simple or complex, or conservative or
liberal. Then, participants judged the issue and policy on different but
related items. Finally, participants filled out a series of surveys
evaluating their current emotional state, conservativeness, personality
traits, need for cognition, emotional regulation, and demographics.

\section{Results}\label{results}

We used R (Version 3.6.1; R Core Team, 2019) and the R-packages
\emph{dplyr} (Version 0.8.3; Wickham, François, Henry, \& Müller, 2019),
\emph{forcats} (Version 0.4.0; Wickham, 2019a), \emph{ggplot2} (Version
3.2.1; Wickham, 2016), \emph{here} (Version 0.1; Müller, 2017),
\emph{papaja} (Version 0.1.0.9842; Aust \& Barth, 2018), \emph{purrr}
(Version 0.3.3; Henry \& Wickham, 2019), \emph{readr} (Version 1.3.1;
Wickham, Hester, \& Francois, 2018), \emph{rio} (Version 0.5.16; C.-h.
Chan, Chan, Leeper, \& Becker, 2018), \emph{stringr} (Version 1.4.0;
Wickham, 2019b), \emph{tibble} (Version 2.1.3; Müller \& Wickham, 2019),
\emph{tidyr} (Version 1.0.0; Wickham \& Henry, 2019), and
\emph{tidyverse} (Version 1.2.1; Wickham, 2017) for all our analyses.

\section{Discussion}\label{discussion}

\newpage

\section{References}\label{references}

\begingroup
\setlength{\parindent}{-0.5in} \setlength{\leftskip}{0.5in}

\hypertarget{refs}{}
\hypertarget{ref-R-papaja}{}
Aust, F., \& Barth, M. (2018). \emph{papaja: Create APA manuscripts with
R Markdown}. Retrieved from \url{https://github.com/crsh/papaja}

\hypertarget{ref-R-rio}{}
Chan, C.-h., Chan, G. C., Leeper, T. J., \& Becker, J. (2018).
\emph{Rio: A swiss-army knife for data file i/o}.

\hypertarget{ref-R-purrr}{}
Henry, L., \& Wickham, H. (2019). \emph{Purrr: Functional programming
tools}. Retrieved from \url{https://CRAN.R-project.org/package=purrr}

\hypertarget{ref-R-here}{}
Müller, K. (2017). \emph{Here: A simpler way to find your files}.
Retrieved from \url{https://CRAN.R-project.org/package=here}

\hypertarget{ref-R-tibble}{}
Müller, K., \& Wickham, H. (2019). \emph{Tibble: Simple data frames}.
Retrieved from \url{https://CRAN.R-project.org/package=tibble}

\hypertarget{ref-R-base}{}
R Core Team. (2019). \emph{R: A language and environment for statistical
computing}. Vienna, Austria: R Foundation for Statistical Computing.
Retrieved from \url{https://www.R-project.org/}

\hypertarget{ref-R-ggplot2}{}
Wickham, H. (2016). \emph{Ggplot2: Elegant graphics for data analysis}.
Springer-Verlag New York. Retrieved from
\url{https://ggplot2.tidyverse.org}

\hypertarget{ref-R-tidyverse}{}
Wickham, H. (2017). \emph{Tidyverse: Easily install and load the
'tidyverse'}. Retrieved from
\url{https://CRAN.R-project.org/package=tidyverse}

\hypertarget{ref-R-forcats}{}
Wickham, H. (2019a). \emph{Forcats: Tools for working with categorical
variables (factors)}. Retrieved from
\url{https://CRAN.R-project.org/package=forcats}

\hypertarget{ref-R-stringr}{}
Wickham, H. (2019b). \emph{Stringr: Simple, consistent wrappers for
common string operations}. Retrieved from
\url{https://CRAN.R-project.org/package=stringr}

\hypertarget{ref-R-tidyr}{}
Wickham, H., \& Henry, L. (2019). \emph{Tidyr: Tidy messy data}.
Retrieved from \url{https://CRAN.R-project.org/package=tidyr}

\hypertarget{ref-R-dplyr}{}
Wickham, H., François, R., Henry, L., \& Müller, K. (2019). \emph{Dplyr:
A grammar of data manipulation}. Retrieved from
\url{https://CRAN.R-project.org/package=dplyr}

\hypertarget{ref-R-readr}{}
Wickham, H., Hester, J., \& Francois, R. (2018). \emph{Readr: Read
rectangular text data}. Retrieved from
\url{https://CRAN.R-project.org/package=readr}

\endgroup


\end{document}
