\documentclass[man,draftall]{apa6}
\usepackage{lmodern}
\usepackage{amssymb,amsmath}
\usepackage{ifxetex,ifluatex}
\usepackage{fixltx2e} % provides \textsubscript
\ifnum 0\ifxetex 1\fi\ifluatex 1\fi=0 % if pdftex
  \usepackage[T1]{fontenc}
  \usepackage[utf8]{inputenc}
\else % if luatex or xelatex
  \ifxetex
    \usepackage{mathspec}
  \else
    \usepackage{fontspec}
  \fi
  \defaultfontfeatures{Ligatures=TeX,Scale=MatchLowercase}
\fi
% use upquote if available, for straight quotes in verbatim environments
\IfFileExists{upquote.sty}{\usepackage{upquote}}{}
% use microtype if available
\IfFileExists{microtype.sty}{%
\usepackage{microtype}
\UseMicrotypeSet[protrusion]{basicmath} % disable protrusion for tt fonts
}{}
\usepackage{hyperref}
\hypersetup{unicode=true,
            pdftitle={The differential effect of anger and fear on political judgments},
            pdfauthor={Sarah Dimakis},
            pdfborder={0 0 0},
            breaklinks=true}
\urlstyle{same}  % don't use monospace font for urls
\usepackage{longtable,booktabs}
\usepackage{graphicx,grffile}
\makeatletter
\def\maxwidth{\ifdim\Gin@nat@width>\linewidth\linewidth\else\Gin@nat@width\fi}
\def\maxheight{\ifdim\Gin@nat@height>\textheight\textheight\else\Gin@nat@height\fi}
\makeatother
% Scale images if necessary, so that they will not overflow the page
% margins by default, and it is still possible to overwrite the defaults
% using explicit options in \includegraphics[width, height, ...]{}
\setkeys{Gin}{width=\maxwidth,height=\maxheight,keepaspectratio}
\IfFileExists{parskip.sty}{%
\usepackage{parskip}
}{% else
\setlength{\parindent}{0pt}
\setlength{\parskip}{6pt plus 2pt minus 1pt}
}
\setlength{\emergencystretch}{3em}  % prevent overfull lines
\providecommand{\tightlist}{%
  \setlength{\itemsep}{0pt}\setlength{\parskip}{0pt}}
\setcounter{secnumdepth}{0}
% Redefines (sub)paragraphs to behave more like sections
\ifx\paragraph\undefined\else
\let\oldparagraph\paragraph
\renewcommand{\paragraph}[1]{\oldparagraph{#1}\mbox{}}
\fi
\ifx\subparagraph\undefined\else
\let\oldsubparagraph\subparagraph
\renewcommand{\subparagraph}[1]{\oldsubparagraph{#1}\mbox{}}
\fi

%%% Use protect on footnotes to avoid problems with footnotes in titles
\let\rmarkdownfootnote\footnote%
\def\footnote{\protect\rmarkdownfootnote}


  \title{The differential effect of anger and fear on political judgments}
    \author{Sarah Dimakis\textsuperscript{1}}
    \date{}
  
\shorttitle{ANGER AND FEAR ON POLITICAL JUDGMENTS}
\affiliation{
\vspace{0.5cm}
\textsuperscript{1} Univeristy of Oregon}
\usepackage{csquotes}
\usepackage{upgreek}
\captionsetup{font=singlespacing,justification=justified}

\usepackage{longtable}
\usepackage{lscape}
\usepackage{multirow}
\usepackage{tabularx}
\usepackage[flushleft]{threeparttable}
\usepackage{threeparttablex}

\newenvironment{lltable}{\begin{landscape}\begin{center}\begin{ThreePartTable}}{\end{ThreePartTable}\end{center}\end{landscape}}

\makeatletter
\newcommand\LastLTentrywidth{1em}
\newlength\longtablewidth
\setlength{\longtablewidth}{1in}
\newcommand{\getlongtablewidth}{\begingroup \ifcsname LT@\roman{LT@tables}\endcsname \global\longtablewidth=0pt \renewcommand{\LT@entry}[2]{\global\advance\longtablewidth by ##2\relax\gdef\LastLTentrywidth{##2}}\@nameuse{LT@\roman{LT@tables}} \fi \endgroup}


\DeclareDelayedFloatFlavor{ThreePartTable}{table}
\DeclareDelayedFloatFlavor{lltable}{table}
\DeclareDelayedFloatFlavor*{longtable}{table}
\makeatletter
\renewcommand{\efloat@iwrite}[1]{\immediate\expandafter\protected@write\csname efloat@post#1\endcsname{}}
\makeatother

\authornote{

Correspondence concerning this article should be addressed to Sarah
Dimakis, University of Oregon Department of Psychology. E-mail:
\href{mailto:sdimakis@uoregon.edu}{\nolinkurl{sdimakis@uoregon.edu}}}

\abstract{
Insert abstract here.


}

\begin{document}
\maketitle

\section{Method}\label{method}

\subsection{Participants}\label{participants}

Two-hundred U.S. adults were recruited via Amazon Mechanical Turk to
participate in a study listed as \enquote{Psychology and society
survey.} Participants were removed for completing the survey quicker
than predetermined time cutoffs or failing to pass attention checks,
leaving 134 participants for analyses. Of the remaining 134
participants, who were 20 to 68 years old (\emph{M} = 40.35, \emph{SD} =
11.99), 53.7\% identified as female and 46.3\% identified as male, while
77.6\% identified as White of Caucasian, 7.5\% as African American or
Black, 6\% as Hispanic or Latinx, 5.2\% as Asian or Asian American,
0.7\% as Native American, and 9.7\% as \enquote{Other.} The sample
leaned liberal, with 44\% affiliated with the Democratic Party, 23.1\%
affiliated with the Republican Party, and 9.7\% affiliated with the
Green Party, Libertarian Party, or no party. The participants were
compensated \$1.50 for completing the survey, which took on average less
than 12 minutes.

\subsection{Materials}\label{materials}

\textbf{Emotion induction.} Affect was induced incidentally through
video clips (Schaefer, Nils, Sanchez, \& Philippot, 2010). Six videos
were played for participants from three affect categories in order to
increase generalizability. Those in the control condition saw one of two
clips from Blue, a typical car ride or a man shuffling papers around in
an office. The two clips intending to elicit primarily fear were from
the Shining (i.e., the protagonists are escaping an axe murderer) and
the Blair Witch Project (i.e., the protagonists are frantically
searching through a house in the dark), while the clips intending to
elicit primarily anger were from Seven (i.e., a serial killer torments a
cop by murdering his wife) and Schindler's List (i.e., a Nazi murders in
a concentration camp).

\textbf{Issue and policy.} The participants read a short speech about
the prevalence of homelessness in the United States (see Appendix A).
The speech presented facts about the increasing number of homeless
people in the country, reasoning that the problem is important and worth
addressing. The participants also read a policy addressing homelessness
(see Appendix A). The policy was either complex or simple, and either
conservative or liberal leaning. The complex policies were comprised of
simple policies, such that a liberal, complex policy proposed three
simple policies that would be implemented concurrently. Half of the
policies were taken from prominent liberal websites and half were taken
from prominent conservative websites.

\textbf{Policy and issue judgments.} Participants rated ten items on a
scale of 1 = strongly disagree to 7 = strongly agree. The items were
judgments about the policy, such as \enquote{The policy will solve the
problem}, and judgments about the issue itself, such as
\enquote{Homelessness is an important issue} (see Appendix B). The items
mainly covered varying facets of policy beliefs, including if the policy
was good, reasonable, possible, effective, and moral.

\textbf{Individual difference measures.} Participants filled out several
individual difference surveys. An adjusted Positive and Negative Affect
Schedule (Watson, Clark, \& Tellegen, 1988) measured the extent to which
participants felt interested, distressed, afraid, guilty, angry,
disgusted, happy, sad, anxious, calm, surprised, and upset. The 12-point
Social and Economic Conservatism Scale (Everett, 2013) measured the
extent to which participants felt positively about conservative issues,
such as gun ownership or limited government. The 10-item Personality
Inventory (Gosling, Rentfrow, \& Swann Jr, 2003) measured the
personality traits extraversion, agreeableness, conscientiousness,
emotional stability, and openness to experiences. The 18-point Need for
Cognition Scale (Cacioppo, Petty, \& Feng Kao, 1984) measured the
tendency and enthusiasm for thinking complexly and abstractly. And last,
the 10-point Emotional Regulation Questionnaire (Gross \& John, 2003)
measured the propensity to regulate emotions through cognitive
reappraisal and expressive suppression techniques. Gender, age,
ethnicity, political orientation, and political affiliation were also
collected.

\subsection{Procedure}\label{procedure}

Participants were redirected to complete the online survey through the
Amazon Mechanical Turk website. Two thirds of participants were randomly
assigned to watch an emotionally evocative film clip, and the remaining
third were randomly assigned to the control condition, where they
watched a neutral film clip. The emotionally evocative clips had been
reliability found to induce predominantly fear or predominantly anger.
After watching a film clip, participants read a short speech about the
prevalence of homelessness in the United States. Then, participants were
randomly assigned to read a policy addressing homelessness that was
either simple or complex, or conservative or liberal. Then, participants
judged the issue and policy on different but related items. Finally,
participants filled out a series of surveys evaluating their current
emotional state, conservativeness, personality traits, need for
cognition, emotional regulation, and demographics.

\section{Results}\label{results}

\begin{longtable}[]{@{}lrrr@{}}
\toprule
& Component 1 & Component 2 & Component 3\tabularnewline
\midrule
\endhead
Interested & 0.15 & -0.10 & 0.84\tabularnewline
Distressed & 0.49 & 0.64 & -0.20\tabularnewline
Afraid & 0.44 & 0.66 & -0.14\tabularnewline
Guilty & 0.70 & 0.20 & -0.02\tabularnewline
Angry & 0.82 & 0.16 & -0.13\tabularnewline
Disgusted & 0.79 & 0.20 & -0.18\tabularnewline
Happy & -0.40 & 0.12 & 0.79\tabularnewline
Sad & 0.80 & 0.29 & -0.06\tabularnewline
Anxious & 0.36 & 0.78 & -0.19\tabularnewline
Calm & -0.26 & -0.38 & 0.69\tabularnewline
Surprised & 0.07 & 0.75 & 0.04\tabularnewline
Upset & 0.81 & 0.36 & -0.13\tabularnewline
\bottomrule
\end{longtable}

\begin{longtable}[]{@{}llll@{}}
\toprule
& Component 1 & Component 2 & Component 3\tabularnewline
\midrule
\endhead
Interested & 0.15 & -0.1 & \textbf{0.84}\tabularnewline
Distressed & 0.49 & \textbf{0.64} & -0.2\tabularnewline
Afraid & 0.44 & \textbf{0.66} & -0.14\tabularnewline
Guilty & \textbf{0.7} & 0.2 & -0.02\tabularnewline
Angry & \textbf{0.82} & 0.16 & -0.13\tabularnewline
Disgusted & \textbf{0.79} & 0.2 & -0.18\tabularnewline
Happy & -0.4 & 0.12 & \textbf{0.79}\tabularnewline
Sad & \textbf{0.8} & 0.29 & -0.06\tabularnewline
Anxious & 0.36 & \textbf{0.78} & -0.19\tabularnewline
Calm & -0.26 & -0.38 & \textbf{0.69}\tabularnewline
Surprised & 0.07 & \textbf{0.75} & 0.04\tabularnewline
Upset & \textbf{0.81} & 0.36 & -0.13\tabularnewline
\bottomrule
\end{longtable}

Emotion factors

We ran principal components on the emotions in Manipulation check

\section{Discussion}\label{discussion}

\newpage

\section{References}\label{references}

\begingroup
\setlength{\parindent}{-0.5in} \setlength{\leftskip}{0.5in}

\hypertarget{refs}{}
\hypertarget{ref-cacioppo1984}{}
Cacioppo, J. T., Petty, R. E., \& Feng Kao, C. (1984). The efficient
assessment of need for cognition. \emph{Journal of Personality
Assessment}, \emph{48}(3), 306--307.

\hypertarget{ref-everett2013}{}
Everett, J. A. (2013). The 12 item social and economic conservatism
scale (secs). \emph{PloS One}, \emph{8}(12), e82131.

\hypertarget{ref-gosling2003}{}
Gosling, S. D., Rentfrow, P. J., \& Swann Jr, W. B. (2003). A very brief
measure of the big-five personality domains. \emph{Journal of Research
in Personality}, \emph{37}(6), 504--528.

\hypertarget{ref-gross2003}{}
Gross, J. J., \& John, O. P. (2003). Individual differences in two
emotion regulation processes: Implications for affect, relationships,
and well-being. \emph{Journal of Personality and Social Psychology},
\emph{85}(2), 348.

\hypertarget{ref-schaefer2010}{}
Schaefer, A., Nils, F., Sanchez, X., \& Philippot, P. (2010). Assessing
the effectiveness of a large database of emotion-eliciting films: A new
tool for emotion researchers. \emph{Cognition and Emotion},
\emph{24}(7), 1153--1172.

\hypertarget{ref-watson1988}{}
Watson, D., Clark, L. A., \& Tellegen, A. (1988). Development and
validation of brief measures of positive and negative affect: The panas
scales. \emph{Journal of Personality and Social Psychology},
\emph{54}(6), 1063.

\endgroup


\end{document}
